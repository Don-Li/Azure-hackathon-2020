\documentclass[]{article}
\usepackage{lmodern}
\usepackage{amssymb,amsmath}
\usepackage{ifxetex,ifluatex}
\usepackage{fixltx2e} % provides \textsubscript
\ifnum 0\ifxetex 1\fi\ifluatex 1\fi=0 % if pdftex
  \usepackage[T1]{fontenc}
  \usepackage[utf8]{inputenc}
\else % if luatex or xelatex
  \ifxetex
    \usepackage{mathspec}
  \else
    \usepackage{fontspec}
  \fi
  \defaultfontfeatures{Ligatures=TeX,Scale=MatchLowercase}
\fi
% use upquote if available, for straight quotes in verbatim environments
\IfFileExists{upquote.sty}{\usepackage{upquote}}{}
% use microtype if available
\IfFileExists{microtype.sty}{%
\usepackage{microtype}
\UseMicrotypeSet[protrusion]{basicmath} % disable protrusion for tt fonts
}{}
\usepackage[margin=1in]{geometry}
\usepackage{hyperref}
\hypersetup{unicode=true,
            pdftitle={deriving\_landmarks},
            pdfauthor={Don Li},
            pdfborder={0 0 0},
            breaklinks=true}
\urlstyle{same}  % don't use monospace font for urls
\usepackage{color}
\usepackage{fancyvrb}
\newcommand{\VerbBar}{|}
\newcommand{\VERB}{\Verb[commandchars=\\\{\}]}
\DefineVerbatimEnvironment{Highlighting}{Verbatim}{commandchars=\\\{\}}
% Add ',fontsize=\small' for more characters per line
\usepackage{framed}
\definecolor{shadecolor}{RGB}{248,248,248}
\newenvironment{Shaded}{\begin{snugshade}}{\end{snugshade}}
\newcommand{\AlertTok}[1]{\textcolor[rgb]{0.94,0.16,0.16}{#1}}
\newcommand{\AnnotationTok}[1]{\textcolor[rgb]{0.56,0.35,0.01}{\textbf{\textit{#1}}}}
\newcommand{\AttributeTok}[1]{\textcolor[rgb]{0.77,0.63,0.00}{#1}}
\newcommand{\BaseNTok}[1]{\textcolor[rgb]{0.00,0.00,0.81}{#1}}
\newcommand{\BuiltInTok}[1]{#1}
\newcommand{\CharTok}[1]{\textcolor[rgb]{0.31,0.60,0.02}{#1}}
\newcommand{\CommentTok}[1]{\textcolor[rgb]{0.56,0.35,0.01}{\textit{#1}}}
\newcommand{\CommentVarTok}[1]{\textcolor[rgb]{0.56,0.35,0.01}{\textbf{\textit{#1}}}}
\newcommand{\ConstantTok}[1]{\textcolor[rgb]{0.00,0.00,0.00}{#1}}
\newcommand{\ControlFlowTok}[1]{\textcolor[rgb]{0.13,0.29,0.53}{\textbf{#1}}}
\newcommand{\DataTypeTok}[1]{\textcolor[rgb]{0.13,0.29,0.53}{#1}}
\newcommand{\DecValTok}[1]{\textcolor[rgb]{0.00,0.00,0.81}{#1}}
\newcommand{\DocumentationTok}[1]{\textcolor[rgb]{0.56,0.35,0.01}{\textbf{\textit{#1}}}}
\newcommand{\ErrorTok}[1]{\textcolor[rgb]{0.64,0.00,0.00}{\textbf{#1}}}
\newcommand{\ExtensionTok}[1]{#1}
\newcommand{\FloatTok}[1]{\textcolor[rgb]{0.00,0.00,0.81}{#1}}
\newcommand{\FunctionTok}[1]{\textcolor[rgb]{0.00,0.00,0.00}{#1}}
\newcommand{\ImportTok}[1]{#1}
\newcommand{\InformationTok}[1]{\textcolor[rgb]{0.56,0.35,0.01}{\textbf{\textit{#1}}}}
\newcommand{\KeywordTok}[1]{\textcolor[rgb]{0.13,0.29,0.53}{\textbf{#1}}}
\newcommand{\NormalTok}[1]{#1}
\newcommand{\OperatorTok}[1]{\textcolor[rgb]{0.81,0.36,0.00}{\textbf{#1}}}
\newcommand{\OtherTok}[1]{\textcolor[rgb]{0.56,0.35,0.01}{#1}}
\newcommand{\PreprocessorTok}[1]{\textcolor[rgb]{0.56,0.35,0.01}{\textit{#1}}}
\newcommand{\RegionMarkerTok}[1]{#1}
\newcommand{\SpecialCharTok}[1]{\textcolor[rgb]{0.00,0.00,0.00}{#1}}
\newcommand{\SpecialStringTok}[1]{\textcolor[rgb]{0.31,0.60,0.02}{#1}}
\newcommand{\StringTok}[1]{\textcolor[rgb]{0.31,0.60,0.02}{#1}}
\newcommand{\VariableTok}[1]{\textcolor[rgb]{0.00,0.00,0.00}{#1}}
\newcommand{\VerbatimStringTok}[1]{\textcolor[rgb]{0.31,0.60,0.02}{#1}}
\newcommand{\WarningTok}[1]{\textcolor[rgb]{0.56,0.35,0.01}{\textbf{\textit{#1}}}}
\usepackage{graphicx}
% grffile has become a legacy package: https://ctan.org/pkg/grffile
\IfFileExists{grffile.sty}{%
\usepackage{grffile}
}{}
\makeatletter
\def\maxwidth{\ifdim\Gin@nat@width>\linewidth\linewidth\else\Gin@nat@width\fi}
\def\maxheight{\ifdim\Gin@nat@height>\textheight\textheight\else\Gin@nat@height\fi}
\makeatother
% Scale images if necessary, so that they will not overflow the page
% margins by default, and it is still possible to overwrite the defaults
% using explicit options in \includegraphics[width, height, ...]{}
\setkeys{Gin}{width=\maxwidth,height=\maxheight,keepaspectratio}
\IfFileExists{parskip.sty}{%
\usepackage{parskip}
}{% else
\setlength{\parindent}{0pt}
\setlength{\parskip}{6pt plus 2pt minus 1pt}
}
\setlength{\emergencystretch}{3em}  % prevent overfull lines
\providecommand{\tightlist}{%
  \setlength{\itemsep}{0pt}\setlength{\parskip}{0pt}}
\setcounter{secnumdepth}{0}
% Redefines (sub)paragraphs to behave more like sections
\ifx\paragraph\undefined\else
\let\oldparagraph\paragraph
\renewcommand{\paragraph}[1]{\oldparagraph{#1}\mbox{}}
\fi
\ifx\subparagraph\undefined\else
\let\oldsubparagraph\subparagraph
\renewcommand{\subparagraph}[1]{\oldsubparagraph{#1}\mbox{}}
\fi

%%% Use protect on footnotes to avoid problems with footnotes in titles
\let\rmarkdownfootnote\footnote%
\def\footnote{\protect\rmarkdownfootnote}

%%% Change title format to be more compact
\usepackage{titling}

% Create subtitle command for use in maketitle
\providecommand{\subtitle}[1]{
  \posttitle{
    \begin{center}\large#1\end{center}
    }
}

\setlength{\droptitle}{-2em}

  \title{deriving\_landmarks}
    \pretitle{\vspace{\droptitle}\centering\huge}
  \posttitle{\par}
    \author{Don Li}
    \preauthor{\centering\large\emph}
  \postauthor{\par}
      \predate{\centering\large\emph}
  \postdate{\par}
    \date{06/06/2020}


\begin{document}
\maketitle

\hypertarget{purpose}{%
\section{Purpose}\label{purpose}}

Derive landmark locations and use this to group the data better.

Approaches:

\begin{itemize}
\tightlist
\item
  Use Google Places API or something like that to determine what
  locations people are going to
\item
  Use statistics
\end{itemize}

I will use the second one because I don't know how to make an API work.

\hypertarget{data-and-stuff}{%
\section{Data and stuff}\label{data-and-stuff}}

\hypertarget{k-means}{%
\subsection{K-means}\label{k-means}}

Use k-means clustering (Hartigan-Wong) on the start and end points.

\begin{Shaded}
\begin{Highlighting}[]
\KeywordTok{load}\NormalTok{( }\StringTok{"G:/azure_hackathon/dataset/trip_summary2_loopytrim.RData"}\NormalTok{ )}

\NormalTok{predict_kmeans =}\StringTok{ }\ControlFlowTok{function}\NormalTok{(object, newdata)\{}
\NormalTok{    centers =}\StringTok{ }\NormalTok{object}\OperatorTok{$}\NormalTok{centers}
\NormalTok{    n_centers =}\StringTok{ }\KeywordTok{nrow}\NormalTok{(centers)}
\NormalTok{    dist_mat =}\StringTok{ }\KeywordTok{as.matrix}\NormalTok{(}\KeywordTok{dist}\NormalTok{(}\KeywordTok{rbind}\NormalTok{(centers, newdata)))}
\NormalTok{    dist_mat =}\StringTok{ }\NormalTok{dist_mat[}\OperatorTok{-}\KeywordTok{seq}\NormalTok{(n_centers), }\KeywordTok{seq}\NormalTok{(n_centers)]}
    \KeywordTok{max.col}\NormalTok{(}\OperatorTok{-}\NormalTok{dist_mat)}
\NormalTok{\}}

\NormalTok{start_end_data =}\StringTok{ }\NormalTok{trip_summary[ , }
    \KeywordTok{list}\NormalTok{( }
        \DataTypeTok{lat1 =}\NormalTok{ start_y, }\DataTypeTok{lng1 =}\NormalTok{ start_x,}
        \DataTypeTok{latN =}\NormalTok{ end_x, }\DataTypeTok{lngN =}\NormalTok{ end_y }
\NormalTok{    ), by =}\StringTok{ "trj_id"}\NormalTok{ ]}

\NormalTok{kmeans_data =}\StringTok{ }\NormalTok{start_end_data[ , \{}
    \KeywordTok{cbind}\NormalTok{( }
        \DataTypeTok{lats =} \KeywordTok{c}\NormalTok{( lat1, latN), }
        \DataTypeTok{lngs =} \KeywordTok{c}\NormalTok{(lng1, lngN )}
\NormalTok{    ) \} ]}

\NormalTok{kmeans_result =}\StringTok{ }\KeywordTok{kmeans}\NormalTok{( kmeans_data, }\DataTypeTok{centers =} \DecValTok{300}\NormalTok{, }\DataTypeTok{nstart =} \DecValTok{1000}\NormalTok{, }\DataTypeTok{iter.max =} \FloatTok{1e9}\NormalTok{ )}

\KeywordTok{save}\NormalTok{( kmeans_result, start_end_data, kmeans_data, }
    \DataTypeTok{file =} \StringTok{"G:/azure_hackathon/dataset/landmarks/landmarks_kmeans.RData"}\NormalTok{ )}
\end{Highlighting}
\end{Shaded}

Put these points on a map

\begin{verbatim}
## sleptTotal= 0
\end{verbatim}

\begin{verbatim}
## [1] "Caution: map type is OpenStreetMap. Until we find the correct projection algorithm, we treat lat/lon like planar coordinates and set TrueProj = FALSE."
\end{verbatim}

\begin{verbatim}
## NULL
\end{verbatim}

\begin{verbatim}
## NULL
\end{verbatim}

\includegraphics{deriving_landmarks_files/figure-latex/unnamed-chunk-2-1.pdf}

\hypertarget{k-means-with-outliers-removed}{%
\subsection{K-means with outliers
removed}\label{k-means-with-outliers-removed}}

I want to automatically detect high transit destinations. In order to do
this, I will filter the points that have no neighbours within 500m.

\begin{Shaded}
\begin{Highlighting}[]
\NormalTok{min_dist_km =}\StringTok{ }\FloatTok{0.5}
\NormalTok{isolated_pts =}\StringTok{ }\KeywordTok{rep}\NormalTok{( F, }\KeywordTok{nrow}\NormalTok{( kmeans_data ) )}
\ControlFlowTok{for}\NormalTok{ ( i }\ControlFlowTok{in} \DecValTok{1}\OperatorTok{:}\KeywordTok{length}\NormalTok{(isolated_pts) )\{}
    \ControlFlowTok{if}\NormalTok{ ( i }\OperatorTok\StringTok{ }\DecValTok{100} \OperatorTok{==}\StringTok{ }\DecValTok{0}\NormalTok{ ) }\KeywordTok{print}\NormalTok{( i )}
\NormalTok{    all_haversines =}\StringTok{ }\KeywordTok{haversine2}\NormalTok{( kmeans_data[i,,}\DataTypeTok{drop =}\NormalTok{ F], kmeans_data )}
\NormalTok{    isolated =}\StringTok{ }\NormalTok{(}\KeywordTok{sum}\NormalTok{(all_haversines }\OperatorTok{<}\StringTok{ }\NormalTok{min_dist_km) }\OperatorTok{-}\StringTok{ }\DecValTok{1}\NormalTok{ ) }\OperatorTok{==}\StringTok{ }\DecValTok{0}
    \ControlFlowTok{if}\NormalTok{ ( isolated )\{}
\NormalTok{        isolated_pts[i] =}\StringTok{ }\NormalTok{T}
\NormalTok{    \}}
\NormalTok{\}}

\NormalTok{new_keans_data =}\StringTok{ }\NormalTok{kmeans_data[ }\OperatorTok{!}\NormalTok{isolated_pts, ]}

\KeywordTok{set.seed}\NormalTok{(}\DecValTok{312}\NormalTok{)}
\NormalTok{kmeans_result2 =}\StringTok{ }\KeywordTok{kmeans}\NormalTok{( new_keans_data, }\DataTypeTok{centers =} \DecValTok{300}\NormalTok{, }
    \DataTypeTok{nstart =} \DecValTok{1000}\NormalTok{, }\DataTypeTok{iter.max =} \FloatTok{1e9}\NormalTok{ )}
\KeywordTok{save}\NormalTok{( new_keans_data, kmeans_result2, }
    \DataTypeTok{file =} \StringTok{"G:/azure_hackathon/dataset/landmarks/landmarks_kmeans_outliers_rm.RData"}\NormalTok{ )}
\end{Highlighting}
\end{Shaded}

In my opinion, the clustering is nicer, especially around Jurong Island.

\begin{verbatim}
## [1] "Caution: map type is OpenStreetMap. Until we find the correct projection algorithm, we treat lat/lon like planar coordinates and set TrueProj = FALSE."
\end{verbatim}

\begin{verbatim}
## NULL
\end{verbatim}

\begin{verbatim}
## NULL
\end{verbatim}

\includegraphics{deriving_landmarks_files/figure-latex/unnamed-chunk-4-1.pdf}

Overlay the two clusters to see how they differ. Picture below. Red is
the clustering with the outliers removed. Blue is the clustering with
the outliers in (first clustering). Expect some randomness due to the
k-means clustering. But, the outliers removed tends to make a lot of
random stuff.

\begin{verbatim}
## [1] "Caution: map type is OpenStreetMap. Until we find the correct projection algorithm, we treat lat/lon like planar coordinates and set TrueProj = FALSE."
\end{verbatim}

\begin{verbatim}
## NULL
\end{verbatim}

\begin{verbatim}
## NULL
\end{verbatim}

\includegraphics{deriving_landmarks_files/figure-latex/unnamed-chunk-5-1.pdf}

\hypertarget{stacking-clustering-algorithms}{%
\section{Stacking clustering
algorithms}\label{stacking-clustering-algorithms}}

Using the results from K-means (outliers removed), we can link clusters
together using DBSCAN. This is useful for points like the airport, where
there is a whole collection of points in a row.

The results below show the linkage of the K-means centers using DBSCAN.
In particular, look at the orange points at the airport (labeled ``2''),
also at some points in the city center. We can use this cluster
information to classify trips based on start or end.

\begin{Shaded}
\begin{Highlighting}[]
\KeywordTok{load}\NormalTok{(  }\StringTok{"G:/azure_hackathon/dataset/landmarks/landmarks_kmeans_outliers_rm.RData"}\NormalTok{ )}

\KeywordTok{library}\NormalTok{( dbscan )}
\KeywordTok{set.seed}\NormalTok{(}\DecValTok{42323}\NormalTok{)}
\NormalTok{dbscan_results =}\StringTok{ }\KeywordTok{dbscan}\NormalTok{( kmeans_result2}\OperatorTok{$}\NormalTok{centers, }\DataTypeTok{minPts =} \DecValTok{1}\NormalTok{, }\DataTypeTok{eps =} \FloatTok{0.01}\NormalTok{ )}
\NormalTok{k2_centers =}\StringTok{ }\KeywordTok{data.table}\NormalTok{(}
    \KeywordTok{cbind}\NormalTok{( kmeans_result2}\OperatorTok{$}\NormalTok{centers, }\DataTypeTok{size =}\NormalTok{ kmeans_result2}\OperatorTok{$}\NormalTok{size,}
        \DataTypeTok{cluster =}\NormalTok{ dbscan_results}\OperatorTok{$}\NormalTok{cluster )}
\NormalTok{)}
\NormalTok{k2_centers[ , total_size }\OperatorTok{:}\ErrorTok{=}\StringTok{ }\KeywordTok{sum}\NormalTok{(size), by =}\StringTok{ "cluster"}\NormalTok{ ]}
\NormalTok{big_clusters =}\StringTok{ }\NormalTok{k2_centers[ total_size }\OperatorTok{>}\StringTok{ }\DecValTok{500}\NormalTok{ ]}

\KeywordTok{save}\NormalTok{( dbscan_results, big_clusters,}
    \DataTypeTok{file =} \StringTok{"G:/azure_hackathon/dataset/landmarks/deriving_landmarks_dbscan.RData"}\NormalTok{ )}
\end{Highlighting}
\end{Shaded}

\begin{verbatim}
## [1] "Caution: map type is OpenStreetMap. Until we find the correct projection algorithm, we treat lat/lon like planar coordinates and set TrueProj = FALSE."
\end{verbatim}

\includegraphics{deriving_landmarks_files/figure-latex/unnamed-chunk-7-1.pdf}

Count up the size of each cluster from K-means, summed over the cluster
membership from DBSCAN and plot only the clusters with more than 100
total.

\begin{verbatim}
## [1] "Caution: map type is OpenStreetMap. Until we find the correct projection algorithm, we treat lat/lon like planar coordinates and set TrueProj = FALSE."
\end{verbatim}

\includegraphics{deriving_landmarks_files/figure-latex/unnamed-chunk-8-1.pdf}

\hypertarget{proof-of-concept}{%
\section{Proof of concept}\label{proof-of-concept}}

To show that this was not for nothing, we can look at trips to specific
places.

We can see below that a trip to the airport does take longer than other
trips.

\includegraphics{deriving_landmarks_files/figure-latex/unnamed-chunk-9-1.pdf}

\begin{verbatim}
## NULL
\end{verbatim}

Similarly, trips to the airport also take longer.

\includegraphics{deriving_landmarks_files/figure-latex/unnamed-chunk-10-1.pdf}

\begin{verbatim}
## NULL
\end{verbatim}

We can do things more algorithmically:

\includegraphics{deriving_landmarks_files/figure-latex/unnamed-chunk-11-1.pdf}

\begin{verbatim}
## NULL
\end{verbatim}

\includegraphics{deriving_landmarks_files/figure-latex/unnamed-chunk-11-2.pdf}

\begin{verbatim}
## NULL
\end{verbatim}

\begin{verbatim}
## 
## Call:
## lm(formula = tdiff2 ~ relevel(trip_Factor, ref = "generic") + 
##     relevel(trip_Factor2, ref = "generic"), data = summary_data)
## 
## Residuals:
##     Min      1Q  Median      3Q     Max 
##  -614.0  -213.9   -66.5   123.0 15140.0 
## 
## Coefficients:
##                                            Estimate Std. Error t value Pr(>|t|)
## (Intercept)                                   5.023      3.283   1.530 0.125998
## relevel(trip_Factor, ref = "generic")C110    25.933      7.558   3.431 0.000602
## relevel(trip_Factor, ref = "generic")C150   -17.605      8.718  -2.019 0.043458
## relevel(trip_Factor, ref = "generic")C16     -8.028      9.566  -0.839 0.401348
## relevel(trip_Factor, ref = "generic")C17    -12.774      8.821  -1.448 0.147595
## relevel(trip_Factor, ref = "generic")C41    -21.074      8.193  -2.572 0.010110
## relevel(trip_Factor, ref = "generic")C43   -129.799     10.259 -12.653  < 2e-16
## relevel(trip_Factor, ref = "generic")C9     -42.999      9.284  -4.631 3.65e-06
## relevel(trip_Factor2, ref = "generic")C110   64.916      7.642   8.494  < 2e-16
## relevel(trip_Factor2, ref = "generic")C150  -15.730      8.706  -1.807 0.070810
## relevel(trip_Factor2, ref = "generic")C16   -10.252      9.947  -1.031 0.302698
## relevel(trip_Factor2, ref = "generic")C17     1.889      8.948   0.211 0.832819
## relevel(trip_Factor2, ref = "generic")C41    48.880      8.736   5.595 2.23e-08
## relevel(trip_Factor2, ref = "generic")C43  -113.518      8.143 -13.941  < 2e-16
## relevel(trip_Factor2, ref = "generic")C9    -51.838      9.231  -5.616 1.97e-08
##                                               
## (Intercept)                                   
## relevel(trip_Factor, ref = "generic")C110  ***
## relevel(trip_Factor, ref = "generic")C150  *  
## relevel(trip_Factor, ref = "generic")C16      
## relevel(trip_Factor, ref = "generic")C17      
## relevel(trip_Factor, ref = "generic")C41   *  
## relevel(trip_Factor, ref = "generic")C43   ***
## relevel(trip_Factor, ref = "generic")C9    ***
## relevel(trip_Factor2, ref = "generic")C110 ***
## relevel(trip_Factor2, ref = "generic")C150 .  
## relevel(trip_Factor2, ref = "generic")C16     
## relevel(trip_Factor2, ref = "generic")C17     
## relevel(trip_Factor2, ref = "generic")C41  ***
## relevel(trip_Factor2, ref = "generic")C43  ***
## relevel(trip_Factor2, ref = "generic")C9   ***
## ---
## Signif. codes:  0 '***' 0.001 '**' 0.01 '*' 0.05 '.' 0.1 ' ' 1
## 
## Residual standard error: 340.7 on 27985 degrees of freedom
## Multiple R-squared:  0.01803,    Adjusted R-squared:  0.01754 
## F-statistic: 36.71 on 14 and 27985 DF,  p-value: < 2.2e-16
\end{verbatim}

\hypertarget{conclusion}{%
\section{Conclusion}\label{conclusion}}

We can see that classifying trips could be a useful covariate.


\end{document}
