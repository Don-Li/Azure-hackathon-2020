\documentclass[]{article}
\usepackage{lmodern}
\usepackage{amssymb,amsmath}
\usepackage{ifxetex,ifluatex}
\usepackage{fixltx2e} % provides \textsubscript
\ifnum 0\ifxetex 1\fi\ifluatex 1\fi=0 % if pdftex
  \usepackage[T1]{fontenc}
  \usepackage[utf8]{inputenc}
\else % if luatex or xelatex
  \ifxetex
    \usepackage{mathspec}
  \else
    \usepackage{fontspec}
  \fi
  \defaultfontfeatures{Ligatures=TeX,Scale=MatchLowercase}
\fi
% use upquote if available, for straight quotes in verbatim environments
\IfFileExists{upquote.sty}{\usepackage{upquote}}{}
% use microtype if available
\IfFileExists{microtype.sty}{%
\usepackage{microtype}
\UseMicrotypeSet[protrusion]{basicmath} % disable protrusion for tt fonts
}{}
\usepackage[margin=1in]{geometry}
\usepackage{hyperref}
\hypersetup{unicode=true,
            pdftitle={Trip summaries},
            pdfauthor={Don Li},
            pdfborder={0 0 0},
            breaklinks=true}
\urlstyle{same}  % don't use monospace font for urls
\usepackage{color}
\usepackage{fancyvrb}
\newcommand{\VerbBar}{|}
\newcommand{\VERB}{\Verb[commandchars=\\\{\}]}
\DefineVerbatimEnvironment{Highlighting}{Verbatim}{commandchars=\\\{\}}
% Add ',fontsize=\small' for more characters per line
\usepackage{framed}
\definecolor{shadecolor}{RGB}{248,248,248}
\newenvironment{Shaded}{\begin{snugshade}}{\end{snugshade}}
\newcommand{\AlertTok}[1]{\textcolor[rgb]{0.94,0.16,0.16}{#1}}
\newcommand{\AnnotationTok}[1]{\textcolor[rgb]{0.56,0.35,0.01}{\textbf{\textit{#1}}}}
\newcommand{\AttributeTok}[1]{\textcolor[rgb]{0.77,0.63,0.00}{#1}}
\newcommand{\BaseNTok}[1]{\textcolor[rgb]{0.00,0.00,0.81}{#1}}
\newcommand{\BuiltInTok}[1]{#1}
\newcommand{\CharTok}[1]{\textcolor[rgb]{0.31,0.60,0.02}{#1}}
\newcommand{\CommentTok}[1]{\textcolor[rgb]{0.56,0.35,0.01}{\textit{#1}}}
\newcommand{\CommentVarTok}[1]{\textcolor[rgb]{0.56,0.35,0.01}{\textbf{\textit{#1}}}}
\newcommand{\ConstantTok}[1]{\textcolor[rgb]{0.00,0.00,0.00}{#1}}
\newcommand{\ControlFlowTok}[1]{\textcolor[rgb]{0.13,0.29,0.53}{\textbf{#1}}}
\newcommand{\DataTypeTok}[1]{\textcolor[rgb]{0.13,0.29,0.53}{#1}}
\newcommand{\DecValTok}[1]{\textcolor[rgb]{0.00,0.00,0.81}{#1}}
\newcommand{\DocumentationTok}[1]{\textcolor[rgb]{0.56,0.35,0.01}{\textbf{\textit{#1}}}}
\newcommand{\ErrorTok}[1]{\textcolor[rgb]{0.64,0.00,0.00}{\textbf{#1}}}
\newcommand{\ExtensionTok}[1]{#1}
\newcommand{\FloatTok}[1]{\textcolor[rgb]{0.00,0.00,0.81}{#1}}
\newcommand{\FunctionTok}[1]{\textcolor[rgb]{0.00,0.00,0.00}{#1}}
\newcommand{\ImportTok}[1]{#1}
\newcommand{\InformationTok}[1]{\textcolor[rgb]{0.56,0.35,0.01}{\textbf{\textit{#1}}}}
\newcommand{\KeywordTok}[1]{\textcolor[rgb]{0.13,0.29,0.53}{\textbf{#1}}}
\newcommand{\NormalTok}[1]{#1}
\newcommand{\OperatorTok}[1]{\textcolor[rgb]{0.81,0.36,0.00}{\textbf{#1}}}
\newcommand{\OtherTok}[1]{\textcolor[rgb]{0.56,0.35,0.01}{#1}}
\newcommand{\PreprocessorTok}[1]{\textcolor[rgb]{0.56,0.35,0.01}{\textit{#1}}}
\newcommand{\RegionMarkerTok}[1]{#1}
\newcommand{\SpecialCharTok}[1]{\textcolor[rgb]{0.00,0.00,0.00}{#1}}
\newcommand{\SpecialStringTok}[1]{\textcolor[rgb]{0.31,0.60,0.02}{#1}}
\newcommand{\StringTok}[1]{\textcolor[rgb]{0.31,0.60,0.02}{#1}}
\newcommand{\VariableTok}[1]{\textcolor[rgb]{0.00,0.00,0.00}{#1}}
\newcommand{\VerbatimStringTok}[1]{\textcolor[rgb]{0.31,0.60,0.02}{#1}}
\newcommand{\WarningTok}[1]{\textcolor[rgb]{0.56,0.35,0.01}{\textbf{\textit{#1}}}}
\usepackage{graphicx}
% grffile has become a legacy package: https://ctan.org/pkg/grffile
\IfFileExists{grffile.sty}{%
\usepackage{grffile}
}{}
\makeatletter
\def\maxwidth{\ifdim\Gin@nat@width>\linewidth\linewidth\else\Gin@nat@width\fi}
\def\maxheight{\ifdim\Gin@nat@height>\textheight\textheight\else\Gin@nat@height\fi}
\makeatother
% Scale images if necessary, so that they will not overflow the page
% margins by default, and it is still possible to overwrite the defaults
% using explicit options in \includegraphics[width, height, ...]{}
\setkeys{Gin}{width=\maxwidth,height=\maxheight,keepaspectratio}
\IfFileExists{parskip.sty}{%
\usepackage{parskip}
}{% else
\setlength{\parindent}{0pt}
\setlength{\parskip}{6pt plus 2pt minus 1pt}
}
\setlength{\emergencystretch}{3em}  % prevent overfull lines
\providecommand{\tightlist}{%
  \setlength{\itemsep}{0pt}\setlength{\parskip}{0pt}}
\setcounter{secnumdepth}{0}
% Redefines (sub)paragraphs to behave more like sections
\ifx\paragraph\undefined\else
\let\oldparagraph\paragraph
\renewcommand{\paragraph}[1]{\oldparagraph{#1}\mbox{}}
\fi
\ifx\subparagraph\undefined\else
\let\oldsubparagraph\subparagraph
\renewcommand{\subparagraph}[1]{\oldsubparagraph{#1}\mbox{}}
\fi

%%% Use protect on footnotes to avoid problems with footnotes in titles
\let\rmarkdownfootnote\footnote%
\def\footnote{\protect\rmarkdownfootnote}

%%% Change title format to be more compact
\usepackage{titling}

% Create subtitle command for use in maketitle
\providecommand{\subtitle}[1]{
  \posttitle{
    \begin{center}\large#1\end{center}
    }
}

\setlength{\droptitle}{-2em}

  \title{Trip summaries}
    \pretitle{\vspace{\droptitle}\centering\huge}
  \posttitle{\par}
    \author{Don Li}
    \preauthor{\centering\large\emph}
  \postauthor{\par}
      \predate{\centering\large\emph}
  \postdate{\par}
    \date{12/06/2020}


\begin{document}
\maketitle

\begin{Shaded}
\begin{Highlighting}[]
\KeywordTok{library}\NormalTok{( data.table )}
\end{Highlighting}
\end{Shaded}

\hypertarget{summarise-trip-trajectories}{%
\section{Summarise trip
trajectories}\label{summarise-trip-trajectories}}

List of covariates:

\begin{itemize}
\tightlist
\item
  \texttt{crow\_dist}: Distance as the crow flies (Haversine, km).
  Numeric.
\item
  \texttt{path\_dist}: Path distance (Haversine, km). Numeric.
\item
  \texttt{path\_dist2}: Path distance; longitude and latitude reversed
  (Haversine, km). Numeric.
\item
  \texttt{timediff}: Observed arrival time. Numeric.
\item
  \texttt{start\_x}, \texttt{start\_y}: Journey start longitudes and
  latitudes. Numeric.
\item
  \texttt{end\_x}, \texttt{end\_y}: Journey end longitudes and
  latitudes. Numeric.
\item
  \texttt{weekday}, \texttt{hour}: Day and hour that the trip started.
  Factor; Numeric.
\item
  \texttt{rush\_hour}: Whether trip started during rush hour. Factor.
\item
  \texttt{mean\_speed}, \texttt{var\_speed}: Mean and variance of speed.
  Numeric.
\item
  \texttt{sampling\_rate}, \texttt{sampling\_rate\_var}: GPS sampling
  rate. Numeric.
\end{itemize}

Covariates to be joined later:

\begin{itemize}
\tightlist
\item
  \texttt{azure\_dist}: Path distance from Azure Maps.
\item
  \texttt{OSRM\_dist}: Path distance from OSRM.
\item
  \texttt{trip\_start}, \texttt{trip\_end}: A factor with levels
  \texttt{generic}, \texttt{CX}, \texttt{CY}, etc. These are landmarks.
  \texttt{generic} catches all other points.
\end{itemize}

\begin{Shaded}
\begin{Highlighting}[]
\KeywordTok{source}\NormalTok{( }\StringTok{"G:/azure_hackathon/data/Don2/trip_summary.R"}\NormalTok{ )}
\KeywordTok{source}\NormalTok{( }\StringTok{"G:/azure_hackathon/data/Don2/distance_functions.R"}\NormalTok{ )}
\KeywordTok{load}\NormalTok{( }\StringTok{"G:/azure_hackathon/datasets2/data_processing/all_data3_speed.RData"}\NormalTok{ )}
\NormalTok{trip_summary =}\StringTok{ }\KeywordTok{sumamrise_trips}\NormalTok{( all_data )}
\end{Highlighting}
\end{Shaded}

\hypertarget{combine-external-distancestimes-from-osrm-and-azure-maps}{%
\subsection{Combine external distances/times from OSRM and Azure
Maps}\label{combine-external-distancestimes-from-osrm-and-azure-maps}}

Join Azure and OSRM distances to the summaries.

\begin{Shaded}
\begin{Highlighting}[]
\KeywordTok{load}\NormalTok{( }\StringTok{"G:/azure_hackathon/datasets2/external_paths/external_dist.RData"}\NormalTok{ )}

\NormalTok{trip_summary[ external_distance_summary,}
    \KeywordTok{c}\NormalTok{(}\StringTok{"azure_dist"}\NormalTok{, }\StringTok{"OSRM_dist"}\NormalTok{) }\OperatorTok{:}\ErrorTok{=}\NormalTok{\{}
        \KeywordTok{list}\NormalTok{( i.azure_dist, i.OSMR_dist )}
\NormalTok{    \},}
\NormalTok{    on =}\StringTok{ "trj_id"}\NormalTok{]}

\KeywordTok{save}\NormalTok{( trip_summary, }
    \DataTypeTok{file =} \StringTok{"G:/azure_hackathon/datasets2/trip_summary/trip_summary1_externaldist.RData"}\NormalTok{ )}
\end{Highlighting}
\end{Shaded}

\hypertarget{loopy-trips}{%
\subsection{Loopy trips}\label{loopy-trips}}

A trip where they just loop around the city. Obviously, we cannot
predict the ETA of these kinds of trips using only the origin and the
destination. There are some trips with very long path distances but
short crow distances (top left).

\begin{Shaded}
\begin{Highlighting}[]
\KeywordTok{load}\NormalTok{( }\StringTok{"G:/azure_hackathon/datasets2/trip_summary/trip_summary1_externaldist.RData"}\NormalTok{ )}
\KeywordTok{load}\NormalTok{( }\StringTok{"G:/azure_hackathon/datasets2/data_processing/all_data3_speed.RData"}\NormalTok{ )}

\NormalTok{trip_summary[ , \{}
    \KeywordTok{plot}\NormalTok{( crow_dist, path_dist, }\DataTypeTok{pch =} \DecValTok{16}\NormalTok{ )}
\NormalTok{\} ]}
\end{Highlighting}
\end{Shaded}

\includegraphics{trip_summaries_files/figure-latex/unnamed-chunk-4-1.pdf}

\begin{verbatim}
## NULL
\end{verbatim}

\begin{Shaded}
\begin{Highlighting}[]
\NormalTok{loopiest_trip =}\StringTok{ }\NormalTok{trip_summary[ }\KeywordTok{which.max}\NormalTok{(path_dist }\OperatorTok{-}\StringTok{ }\NormalTok{crow_dist) ]}
\NormalTok{all_data[ trj_id }\OperatorTok{==}\StringTok{ }\NormalTok{loopiest_trip}\OperatorTok{$}\NormalTok{trj_id, \{}
    \KeywordTok{plot}\NormalTok{( rawlat, rawlng, }\DataTypeTok{type =} \StringTok{"l"}\NormalTok{ )}
\NormalTok{    \} ]}
\end{Highlighting}
\end{Shaded}

\includegraphics{trip_summaries_files/figure-latex/unnamed-chunk-5-1.pdf}

\begin{verbatim}
## NULL
\end{verbatim}

I thought about taking them out. But these trips contain information
about driver behaviour that deviates from the shortest path distance.
So, I think it is best to keep them in.

\hypertarget{add-landmark-data}{%
\subsection{Add landmark data}\label{add-landmark-data}}

Using the results from our clustering exercise, we have landmarks for
\texttt{trip\_start} and \texttt{trip\_end}. \texttt{generic} is the
factor level where the trip does not start/end from one of our
identified landmarks.

\begin{Shaded}
\begin{Highlighting}[]
\KeywordTok{load}\NormalTok{( }\StringTok{"G:/azure_hackathon/datasets2/landmarks/deriving_landmarks_dbscan.RData"}\NormalTok{ )}
\KeywordTok{add_landmarks}\NormalTok{( trip_summary, big_clusters )}
\end{Highlighting}
\end{Shaded}

\hypertarget{save-the-file}{%
\section{Save the file}\label{save-the-file}}

\begin{Shaded}
\begin{Highlighting}[]
\KeywordTok{save}\NormalTok{( trip_summary,}
    \DataTypeTok{file =} \StringTok{"G:/azure_hackathon/datasets2/trip_summary/trip_summary2_landmark.RData"}\NormalTok{ )}
\end{Highlighting}
\end{Shaded}


\end{document}
